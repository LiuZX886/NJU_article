% --- preamble.tex ---
% 本文件包含了常用的LaTeX宏包和推荐设置。
% 请在主文件 (main.tex) 的导言区使用 % --- preamble.tex ---
% 本文件包含了常用的LaTeX宏包和推荐设置。
% 请在主文件 (main.tex) 的导言区使用 % --- preamble.tex ---
% 本文件包含了常用的LaTeX宏包和推荐设置。
% 请在主文件 (main.tex) 的导言区使用 % --- preamble.tex ---
% 本文件包含了常用的LaTeX宏包和推荐设置。
% 请在主文件 (main.tex) 的导言区使用 \input{preamble.tex} 来加载。

% --- 数学环境宏包 ---
\usepackage{amsmath}        % AMS 数学公式环境,提供 align, gather 等
\usepackage{bm}             % 用于处理数学公式中的粗体 \bm{}

% --- 图形与浮动体宏包 ---
\usepackage{subcaption}     % 支持子图、子表环境 (subfigure, subtable)
\usepackage{float}          % 提供 [H] 浮动体选项,强制将图表固定在当前位置
\usepackage{wrapfig}        % 支持文字环绕图形

% --- 表格工具宏包 ---
\usepackage{booktabs}       % 提供专业的三线表命令 (\toprule, \midrule, \bottomrule)
\usepackage{longtable}      % 支持跨页长表格
\usepackage{multirow}       % 支持表格内的多行合并单元格
% tabularx 宏包已在 njuabstract.sty 中加载,此处无需重复

% --- 排版与列表工具 ---
\usepackage{enumitem}       % 方便地自定义列表环境 (itemize, enumerate)
\usepackage{microtype}      % 改善文本间距和断字的微调排版工具,提升阅读体验
\usepackage{lipsum}         % 用于生成测试用的随机文本 \lipsum[1-10]

% --- 超链接详细设置 ---
\usepackage[
    colorlinks=true,            % 将链接显示为彩色而非方框
    linkcolor=black,            % 内部链接颜色(目录、章节跳转)
    citecolor=blue,             % 参考文献引用颜色
    urlcolor=blue,              % URL链接颜色
    bookmarksopen=true,         % 打开PDF时展开书签
    bookmarksnumbered=true,     % 在书签中显示章节编号
    pdfusetitle                 % 自动、安全地使用 \title 和 \author 作为PDF元数据
]{hyperref}


% --- 其他常用工具 ---
% \usepackage{todonotes}      % 添加待办事项和边注,方便草稿阶段修改

% --- 自定义命令区 ---
% 您可以在这里添加您自己的常用命令
% 示例:
% \newcommand{\R}{\mathbb{R}} % 定义实数集符号 来加载。

% --- 数学环境宏包 ---
\usepackage{amsmath}        % AMS 数学公式环境,提供 align, gather 等
\usepackage{bm}             % 用于处理数学公式中的粗体 \bm{}

% --- 图形与浮动体宏包 ---
\usepackage{subcaption}     % 支持子图、子表环境 (subfigure, subtable)
\usepackage{float}          % 提供 [H] 浮动体选项,强制将图表固定在当前位置
\usepackage{wrapfig}        % 支持文字环绕图形

% --- 表格工具宏包 ---
\usepackage{booktabs}       % 提供专业的三线表命令 (\toprule, \midrule, \bottomrule)
\usepackage{longtable}      % 支持跨页长表格
\usepackage{multirow}       % 支持表格内的多行合并单元格
% tabularx 宏包已在 njuabstract.sty 中加载,此处无需重复

% --- 排版与列表工具 ---
\usepackage{enumitem}       % 方便地自定义列表环境 (itemize, enumerate)
\usepackage{microtype}      % 改善文本间距和断字的微调排版工具,提升阅读体验
\usepackage{lipsum}         % 用于生成测试用的随机文本 \lipsum[1-10]

% --- 超链接详细设置 ---
\usepackage[
    colorlinks=true,            % 将链接显示为彩色而非方框
    linkcolor=black,            % 内部链接颜色(目录、章节跳转)
    citecolor=blue,             % 参考文献引用颜色
    urlcolor=blue,              % URL链接颜色
    bookmarksopen=true,         % 打开PDF时展开书签
    bookmarksnumbered=true,     % 在书签中显示章节编号
    pdfusetitle                 % 自动、安全地使用 \title 和 \author 作为PDF元数据
]{hyperref}


% --- 其他常用工具 ---
% \usepackage{todonotes}      % 添加待办事项和边注,方便草稿阶段修改

% --- 自定义命令区 ---
% 您可以在这里添加您自己的常用命令
% 示例:
% \newcommand{\R}{\mathbb{R}} % 定义实数集符号 来加载。

% --- 数学环境宏包 ---
\usepackage{amsmath}        % AMS 数学公式环境,提供 align, gather 等
\usepackage{bm}             % 用于处理数学公式中的粗体 \bm{}

% --- 图形与浮动体宏包 ---
\usepackage{subcaption}     % 支持子图、子表环境 (subfigure, subtable)
\usepackage{float}          % 提供 [H] 浮动体选项,强制将图表固定在当前位置
\usepackage{wrapfig}        % 支持文字环绕图形

% --- 表格工具宏包 ---
\usepackage{booktabs}       % 提供专业的三线表命令 (\toprule, \midrule, \bottomrule)
\usepackage{longtable}      % 支持跨页长表格
\usepackage{multirow}       % 支持表格内的多行合并单元格
% tabularx 宏包已在 njuabstract.sty 中加载,此处无需重复

% --- 排版与列表工具 ---
\usepackage{enumitem}       % 方便地自定义列表环境 (itemize, enumerate)
\usepackage{microtype}      % 改善文本间距和断字的微调排版工具,提升阅读体验
\usepackage{lipsum}         % 用于生成测试用的随机文本 \lipsum[1-10]

% --- 超链接详细设置 ---
\usepackage[
    colorlinks=true,            % 将链接显示为彩色而非方框
    linkcolor=black,            % 内部链接颜色(目录、章节跳转)
    citecolor=blue,             % 参考文献引用颜色
    urlcolor=blue,              % URL链接颜色
    bookmarksopen=true,         % 打开PDF时展开书签
    bookmarksnumbered=true,     % 在书签中显示章节编号
    pdfusetitle                 % 自动、安全地使用 \title 和 \author 作为PDF元数据
]{hyperref}


% --- 其他常用工具 ---
% \usepackage{todonotes}      % 添加待办事项和边注,方便草稿阶段修改

% --- 自定义命令区 ---
% 您可以在这里添加您自己的常用命令
% 示例:
% \newcommand{\R}{\mathbb{R}} % 定义实数集符号 来加载。

% --- 数学环境宏包 ---
\usepackage{amsmath}        % AMS 数学公式环境,提供 align, gather 等
\usepackage{bm}             % 用于处理数学公式中的粗体 \bm{}

% --- 图形与浮动体宏包 ---
\usepackage{subcaption}     % 支持子图、子表环境 (subfigure, subtable)
\usepackage{float}          % 提供 [H] 浮动体选项,强制将图表固定在当前位置
\usepackage{wrapfig}        % 支持文字环绕图形

% --- 表格工具宏包 ---
\usepackage{booktabs}       % 提供专业的三线表命令 (\toprule, \midrule, \bottomrule)
\usepackage{longtable}      % 支持跨页长表格
\usepackage{multirow}       % 支持表格内的多行合并单元格
% tabularx 宏包已在 njuabstract.sty 中加载,此处无需重复

% --- 排版与列表工具 ---
\usepackage{enumitem}       % 方便地自定义列表环境 (itemize, enumerate)
\usepackage{microtype}      % 改善文本间距和断字的微调排版工具,提升阅读体验
\usepackage{lipsum}         % 用于生成测试用的随机文本 \lipsum[1-10]

% --- 超链接详细设置 ---
\usepackage[
    colorlinks=true,            % 将链接显示为彩色而非方框
    linkcolor=black,            % 内部链接颜色(目录、章节跳转)
    citecolor=blue,             % 参考文献引用颜色
    urlcolor=blue,              % URL链接颜色
    bookmarksopen=true,         % 打开PDF时展开书签
    bookmarksnumbered=true,     % 在书签中显示章节编号
    pdfusetitle                 % 自动、安全地使用 \title 和 \author 作为PDF元数据
]{hyperref}


% --- 其他常用工具 ---
% \usepackage{todonotes}      % 添加待办事项和边注,方便草稿阶段修改

% --- 自定义命令区 ---
% 您可以在这里添加您自己的常用命令
% 示例:
% \newcommand{\R}{\mathbb{R}} % 定义实数集符号